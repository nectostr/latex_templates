%% Автор: Роман Бельтюков
%%
%% Как использовать данный шаблон:
%% В тексте шаблоны определены блоки, начинающиеся с комментария со знаками %%
%% и заканчивающиеся комментарием с одинарным знаком %
%% Внутри каждого такого блока определен текст, который можно (нужно) менять - 
%% будь то название работы, ФИО или еще что-то

\documentclass[12pt,a4paper]{extreport}
\usepackage{cmap}					% поиск в PDF

% Русский язык
\usepackage[T2A]{fontenc}			% кодировка
\usepackage[utf8]{inputenc}			% кодировка исходного текста
\usepackage[english,russian]{babel}	% локализация и переносы

\usepackage{extsizes}	% Возможность использовать 14й шрифт

\usepackage{tabularx}	% Таблица с равной ширины колонками

\usepackage{geometry}	% Поля
\geometry{
	a4paper,
	left=30mm,
	top=2cm,
	right=1cm,
	bottom=2cm
}

\usepackage{soul}		% Начертание (подчеркнутыйй и т.д.)

\usepackage{setspace}	% Интерлиньяж (межстрочный интервал)

\begin{document}
	\pagestyle{empty}
	
	% Заголовок учреждения
	\begin{center}
		{\footnotesize
%% Название гос учреждения
		ФЕДЕРАЛЬНОЕ ГОСУДАРСТВЕННОЕ АВТОНОМНОЕ
		ОБРАЗОВАТЕЛЬНОЕ УЧРЕЖДЕНИЕ ВЫСШЕГО ОБРАЗОВАНИЯ
%
		\\
		\textbf{
%% Еще немного официальных названий
			«САНКТ-ПЕТЕРБУРГСКИЙ ПОЛИТЕХНИЧЕСКИЙ УНИВЕРСИТЕТ \\ ПЕТРА ВЕЛИКОГО» \\
			ИНСТИТУТ КОМПЬЮТЕРНЫХ НАУК И ТЕХНОЛОГИЙ \\
			КАФЕДРА «КОМПЬЮТЕРНЫЕ ИНТЕЛЛЕКТУАЛЬНЫЕ ТЕХНОЛОГИИ»
%
		}
		}
	\end{center}
	\vspace{10pt}
	\vspace{\fill}
	% Название документа
	\begin{center}
		{\Large
		Отчет о прохождении \\
		стационарной производственной практики
		(научно-исследовательская работа магистра на тему: {\large
%% Название работы
			«Исследование и адаптация существующих алгоритмов естественного градиентного спуска к задаче Q-learning»
%
		}) \\
		\vspace{10pt}
%% Автор и группа
		Бельтюкова Романа Константиновича, гр. 23546/1
%
		}
		\vspace{10pt}
	\end{center}


	\begin{flushleft}
	
		% Направление подготовки
		\begin{spacing}{1.2}
			\textbf{Направление подготовки:} 
			{\large
%% Направление подготовки 
				02.04.03 Математическое обеспечение и администрирование информационных систем, \\
				02.04.03\_01 Математическое обеспечение и администрирование корпоративных информационных систем
%
			}
		\end{spacing}
		\vspace{6pt}
			
		% Место прохождения практики
		\textbf{Место прохождения практики:} {\large
%% Место прохождения практики
			СПбПУ, ИКНТ, кафедра «КИТ»
%
		}
		\hrule
		\textit{\centering \footnotesize (указывается наименование профильной организации или наименование структурного подразделения\\}
%% Вторая линия наименования (если в первую не влезло)
		\,
%
		\hrule
		\vspace{0.5mm}
		\textit{\footnotesize {\centering ФГАОУ ВО «СПбПУ», фактический адрес)\\}}
		\vspace{10pt}
		
		% Сроки практики
		\textbf{Сроки практики:} {\large 
%% Сроки практики
			с 03.09.18 по 22.12.18
%
		}
		\vspace{10pt}
		
		% Руководитель
		\textbf{Руководитель практики от ФГАОУ ВО «СПбПУ»:} \\
		{\centering
			{\large
%% Руководитель практики от СПбПУ
				Щукин А.В., доцент кафедры КИТ, к.т.н.
%
				\\
			}\hrule\vspace{0.5mm}
			{\footnotesize \textit{(Ф.И.О., уч. степень, должность)} \\}
		}
		\vspace{10pt}
		
		% Руководитель 2
		\textbf{
%% Название должности консультанта (руководителя от профильной организации)
			Консультант от ФГАОУ ВО «СПбПУ»
%
			:}\\
		{\centering
%% ФИО и должность консультанта (руководителя от профильной организации)
			Тушканова О. Н., доцент кафедры КИТ, к.т.н.
%
			\hrule\vspace{0.5mm}
			{\footnotesize \textit{(Ф.И.О., должность)}}\\
			\,\hrule
		}
		\vspace{15pt}
		
		% Оценка
		\textbf{Оценка: }\rule{19ex}{0.5pt}
		\vspace{20pt}

		% Руководители
		\begin{tabularx}{\linewidth}{@{} XX>{\hsize=.6\hsize}X}
			Руководитель практики \\
			от ФГАОУ ВО «СПбПУ»
			&
			&
%% ФИО руководителя
			Щукин А.В. 
%
			\vspace{10pt} \\
%% Название должности консультанта (руководителя от профильной ляляля)
			Консультант от \\
			ФГАОУ ВО «СПбПУ»
%
			&
			&
%% ФИО консультанта
			Тушканова О.Н.
%
			\vspace{12pt} \\
			Обучающийся
			&
			&
%% Ваше ФИО
			Бельтюков Р.К.
%
		\end{tabularx}
		\vspace{20pt}
		
		% Дата
		Дата: \rule{10ex}{0.5pt}
	\end{flushleft}
\end{document}